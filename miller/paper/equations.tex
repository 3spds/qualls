\documentclass[a4paper,10pt, twocolumn]{report}
\usepackage[utf8]{inputenc}
\usepackage{graphicx}
\usepackage[font=footnotesize]{caption}
\usepackage{mathtools}

% Title Page
\title{Miller Paper}
\author{Joseph Mariglio}



\begin{document}
\maketitle

\begin{abstract}
\end{abstract}

In liquids of low viscosity, stress is the force normal to the surface per unit area:
\begin{equation}\label{stressnorm}
T = \frac{F_n}{Area}
\end{equation}

On solids and viscous liquids, stress is a vector $T^n$ defined by:
\begin{equation}\label{stressvec}
T^n = n  \cdot \sigma
\end{equation}

where $n$ is the vector normal to the surface and $\sigma$ is the Cauchy stress tensor. 
We will simplify our derivation to use the definition in \eqref{stressnorm} with no loss to the derivation.
Stress on the object results in strain, a dimensionless proportion of displacements given by:
\begin{equation}
\frac{l_{s}}{l_{e}}
\end{equation}
where $l_s$ and $l_e$ are the lengths of the material while stressed and at equilibrium, respectively.

The displacement throughout the object will vary continuously throughout the object, from particle to particle.

Of particular interest are the local deformations of the object, rather than the overall translation of the object in space.
These local deformations are given by displacement gradient $\nabla u$, a 2nd rank tensor:
\begin{equation}
\nabla u \begin{pmatrix}
         \frac{\partial u_x}{\partial x} & \frac{\partial u_x}{\partial y} & \frac{\partial u_x}{\partial z} \\[0.5em]
         \frac{\partial u_y}{\partial x} & \frac{\partial u_y}{\partial y} & \frac{\partial u_y}{\partial z} \\[0.5em]
         \frac{\partial u_z}{\partial x} & \frac{\partial u_z}{\partial y} & \frac{\partial u_z}{\partial z}
        \end{pmatrix}
\end{equation}

The strain matrix $S$ eliminates the contributions given by rotations and describes local stretching only. 
The formula for $S$ is:
\begin{equation}
S = \frac{\nabla u + \left(\nabla u \right)^T}{2}
\end{equation}

We are now equipped to derive the equation of motion for an elastic solid.

Plane wave propagating in the x direction:
\begin{equation}\label{planewave}
u(x,y,z,t) = (u_x x + u_y y + u_z z) e^{i(\omega t - k x)}
\end{equation}

\bibliography{/media/joe/Milarepa/Quall_Docs/bibliography/qualls}
\bibliographystyle{plain}
\nocite{*}

\end{document}          
